\documentclass[french]{article}
 
\usepackage[utf8]{inputenc}
\usepackage[T1]{fontenc}
\usepackage{babel}
\usepackage{graphicx}
\usepackage{amsmath} 
\begin{document}

\begin{figure}
	\include{./image/crop_tvitesse=28_volume=003.png}
\end{figure}
La tension de surfaces

La tension de surface ou tension surpificielle ou

La capillarité étudie l'interface entre l'air et un liquide ou entre 2 liquides
non mixibles.

Le mouillage est l'action de mouiller et mouiller consiste à  mettre en contact avec un liquide.

Nous nous interessons en particulier au contact d'une goutte avec un support matériel.

C'est le paramètre d'étalment $S = \gamma_{SG} - \gamma_{SL} - \gamma_{LG}$ qui caractérise le mouillage lorsque la goutte est en équilibre sur un support matériel.

\begin{description}
\item[$S < 0$ :] Mouillage partiel
\item[$S > 0$ :] Mouillage total
\end{description}

Lorsque la goutte est en équilire nous avons:

\[\gamma_{SG}\ = \gamma_{SL} - \gamma_{LG}\cos\theta_{E}\]

Lorsque la goutte est en mouvement, l'angle dynamique  est giiLa capillarité est l'étude de l'interaction en entre 2 liquides non miscinles ou entre un liqude et l'air ou en un liquide et une surface.

La tension de surface 

La capillarité étudie l'interface entre l'air et un liquide ou entre 2 liquides
non mixibles.

Le mouillage est l'action de mouiller et mouiller consiste à mettre en contact avec un liquide.

Nous nous interessons en particulier au contact d'une goutte avec un support matériel.

C'est le paramètre d'étalment $S = \gamma_{SG} - \gamma_{SL} - \gamma_{LG}$ qui caractérise le mouillage lorsque la goutte est en équilibre sur un support matériel.

\begin{description}
\item[$S < 0$ :] Mouillage partiel
\item[$S > 0$ :] Mouillage total
\end{description}

Lorsque la que est en équilire nous avons:

\[\gamma_{SG}\ = \gamma_{SL} - \gamma_{LG}\cos\theta_{E} \]

\begin{figure}
	%LaTeX with PSTricks extensions
%%Creator: inkscape 0.91
%%Please note this file requires PSTricks extensions
\psset{xunit=.5pt,yunit=.5pt,runit=.5pt}
\begin{pspicture}(354.33070866,177.16535433)
{
\newrgbcolor{curcolor}{0 0 0}
\pscustom[linewidth=3.83385825,linecolor=curcolor]
{
\newpath
\moveto(1573.5781,1025.7649667)
\curveto(1307.0076,1026.3441667)(1187.3999,438.9516367)(1842.8574,434.2864467)
\lineto(1164.123,69.6575467)
\curveto(1331.3926,-349.3256633)(1315.108,-352.3800433)(1680,-40.0006633)
\curveto(1910.5475,209.6129847)(16257.144,811.4281967)(1768.5723,922.8567667)
\curveto(1700.9049,995.7697067)(1633.5246,1025.6347167)(1573.5781,1025.7649667)
\closepath
\moveto(1164.123,69.6575467)
\curveto(1162.7263,73.1561267)(1161.4227,76.4426667)(1160,79.9993407)
\lineto(1062.9961,15.3313767)
\lineto(1164.123,69.6575467)
\closepath
\moveto(1062.9961,15.3313767)
\lineto(960,-40.0006633)
\lineto(851.42773,-125.7135533)
\lineto(1062.9961,15.3313767)
\closepath
}
}
{
\newrgbcolor{curcolor}{0 0 0}
\pscustom[linewidth=3.09585879,linecolor=curcolor]
{
\newpath
\moveto(-464.49016,1186.30866216)
\lineto(22.4184872,1186.30866216)
}
}
{
\newrgbcolor{curcolor}{0 0 0}
\pscustom[linewidth=2.12991232,linecolor=curcolor]
{
\newpath
\moveto(-446.4565064,1186.30866216)
\lineto(-464.49016,1145.09844216)
\lineto(-464.49016,1145.09844216)
}
}
{
\newrgbcolor{curcolor}{0 0 0}
\pscustom[linewidth=2.12991232,linecolor=curcolor]
{
\newpath
\moveto(-425.71780476,1186.30866216)
\lineto(-443.75145836,1145.09844216)
\lineto(-443.75145836,1145.09844216)
}
}
{
\newrgbcolor{curcolor}{0 0 0}
\pscustom[linewidth=2.12991232,linecolor=curcolor]
{
\newpath
\moveto(-403.17573776,1186.30866216)
\lineto(-421.20939136,1145.09844216)
\lineto(-421.20939136,1145.09844216)
}
}
{
\newrgbcolor{curcolor}{0 0 0}
\pscustom[linewidth=2.12991232,linecolor=curcolor]
{
\newpath
\moveto(-370.71516128,1186.30866216)
\lineto(-388.74881488,1145.09844216)
\lineto(-388.74881488,1145.09844216)
}
}
{
\newrgbcolor{curcolor}{0.52549022 0.52549022 0.74901962}
\pscustom[linestyle=none,fillstyle=solid,fillcolor=curcolor]
{
\newpath
\moveto(919.9999828,368.76405661)
\lineto(1119.99995377,524.80312792)
\lineto(1119.99995377,-259.14923107)
\lineto(919.9999828,-275.19682425)
\closepath
}
}
{
\newrgbcolor{curcolor}{0.84313726 0.84313726 1}
\pscustom[linestyle=none,fillstyle=solid,fillcolor=curcolor]
{
\newpath
\moveto(1119.99995377,524.80312792)
\lineto(1119.99995377,-259.14923107)
\lineto(558.63977974,-393.35183321)
\lineto(558.63977974,-780.11837508)
\closepath
}
}
{
\newrgbcolor{curcolor}{0.3019608 0.3019608 0.62352943}
\pscustom[linestyle=none,fillstyle=solid,fillcolor=curcolor]
{
\newpath
\moveto(919.9999828,368.76405661)
\lineto(1119.99995377,524.80312792)
\lineto(558.63977974,-780.11837508)
\lineto(625.75234258,-831.90917524)
\closepath
}
}
{
\newrgbcolor{curcolor}{0.20784314 0.20784314 0.39215687}
\pscustom[linestyle=none,fillstyle=solid,fillcolor=curcolor]
{
\newpath
\moveto(919.9999828,368.76405661)
\lineto(919.9999828,-275.19682425)
\lineto(625.75234258,-398.6781767)
\lineto(625.75234258,-831.90917524)
\closepath
}
}
{
\newrgbcolor{curcolor}{0.68627453 0.68627453 0.87058824}
\pscustom[linestyle=none,fillstyle=solid,fillcolor=curcolor]
{
\newpath
\moveto(919.9999828,-275.19682425)
\lineto(1119.99995377,-259.14923107)
\lineto(558.63977974,-393.35183321)
\lineto(625.75234258,-398.6781767)
\closepath
}
}
{
\newrgbcolor{curcolor}{0.9137255 0.9137255 1}
\pscustom[linestyle=none,fillstyle=solid,fillcolor=curcolor]
{
\newpath
\moveto(625.75234258,-831.90917524)
\lineto(558.63977974,-780.11837508)
\lineto(558.63977974,-393.35183321)
\lineto(625.75234258,-398.6781767)
\closepath
}
}
{
\newrgbcolor{curcolor}{0 0 0}
\pscustom[linewidth=1,linecolor=curcolor]
{
\newpath
\moveto(110.23023,79.9640967)
\curveto(77.857143,151.4285667)(250.57436,129.4972267)(280.8876,79.4082667)
}
}
{
\newrgbcolor{curcolor}{0 0 0}
\pscustom[linewidth=0.70866144,linecolor=curcolor]
{
\newpath
\moveto(106.13916319,95.68642317)
\curveto(114.2807126,95.19456308)(121.04646576,88.34256782)(122.45298405,79.16465234)
}
}
{
\newrgbcolor{curcolor}{0 0 0}
\pscustom[linestyle=none,fillstyle=solid,fillcolor=curcolor]
{
\newpath
\moveto(123.67230911,95.40929058)
\lineto(119.54265145,95.40929058)
\curveto(119.60748124,94.18510377)(119.78900465,93.3130242)(120.08722169,92.79305187)
\curveto(120.45891249,92.15434571)(120.96674585,91.83499262)(121.61072177,91.83499262)
\curveto(122.25901967,91.83499262)(122.76253105,92.15639284)(123.12125589,92.79919328)
\curveto(123.43676087,93.36420258)(123.62044527,94.23423501)(123.67230911,95.40929058)
\closepath
\moveto(123.65286017,96.45332951)
\curveto(123.53184456,97.61610227)(123.35464313,98.40220216)(123.12125589,98.81162919)
\curveto(122.74956509,99.45852389)(122.24605371,99.78197125)(121.61072177,99.78197125)
\curveto(120.9494579,99.78197125)(120.44378554,99.46261816)(120.09370467,98.823912)
\curveto(119.81709756,98.30803394)(119.63773514,97.51783978)(119.55561741,96.45332951)
\lineto(123.65286017,96.45332951)
\closepath
\moveto(121.61072177,100.70318206)
\curveto(122.64799842,100.70318206)(123.46269278,100.26918941)(124.05480487,99.40120411)
\curveto(124.64691696,98.53731308)(124.942973,97.33973902)(124.942973,95.80848194)
\curveto(124.942973,94.28131912)(124.64691696,93.08374506)(124.05480487,92.21575976)
\curveto(123.46269278,91.34368019)(122.64799842,90.90764041)(121.61072177,90.90764041)
\curveto(120.56912313,90.90764041)(119.75442877,91.34368019)(119.16663866,92.21575976)
\curveto(118.57452658,93.08374506)(118.27847053,94.28131912)(118.27847053,95.80848194)
\curveto(118.27847053,97.33973902)(118.57452658,98.53731308)(119.16663866,99.40120411)
\curveto(119.75442877,100.26918941)(120.56912313,100.70318206)(121.61072177,100.70318206)
\closepath
}
}
{
\newrgbcolor{curcolor}{0 0 0}
\pscustom[linestyle=none,fillstyle=solid,fillcolor=curcolor]
{
\newpath
\moveto(128.62076722,90.78082041)
\curveto(127.99429531,90.78082041)(127.56025984,90.71295787)(127.31866081,90.5772328)
\curveto(127.07706178,90.44150774)(126.95626226,90.20997674)(126.95626226,89.88263982)
\curveto(126.95626226,89.62183479)(127.04615957,89.41425527)(127.2259542,89.25990127)
\curveto(127.40855812,89.10820855)(127.65577574,89.03236219)(127.96760704,89.03236219)
\curveto(128.39742858,89.03236219)(128.74156673,89.17607109)(129.00002151,89.46348887)
\curveto(129.26128558,89.75356794)(129.39191762,90.13812229)(129.39191762,90.61715194)
\lineto(129.39191762,90.78082041)
\lineto(128.62076722,90.78082041)
\closepath
\moveto(130.16728195,91.08420585)
\lineto(130.16728195,88.53337297)
\lineto(129.39191762,88.53337297)
\lineto(129.39191762,89.21199831)
\curveto(129.21493228,88.94054818)(128.99440293,88.73962185)(128.73032957,88.60921934)
\curveto(128.46625621,88.4814781)(128.14318774,88.41760748)(127.76112415,88.41760748)
\curveto(127.27792609,88.41760748)(126.89305321,88.54534872)(126.60650552,88.80083119)
\curveto(126.32276712,89.05897495)(126.18089793,89.40361017)(126.18089793,89.83473685)
\curveto(126.18089793,90.33771798)(126.35788326,90.71694979)(126.71185394,90.97243226)
\curveto(127.0686339,91.22791474)(127.59958991,91.35565598)(128.30472197,91.35565598)
\lineto(129.39191762,91.35565598)
\lineto(129.39191762,91.42751043)
\curveto(129.39191762,91.76549246)(129.27392739,92.02629749)(129.03794694,92.20992552)
\curveto(128.80477578,92.39621483)(128.47608873,92.48935948)(128.05188578,92.48935948)
\curveto(127.78219383,92.48935948)(127.51952512,92.45875481)(127.26387963,92.39754547)
\curveto(127.00823415,92.33633612)(126.76242118,92.24452211)(126.52644073,92.12210342)
\lineto(126.52644073,92.80072875)
\curveto(126.81017912,92.90451851)(127.08548965,92.98169551)(127.3523723,93.03225975)
\curveto(127.61925495,93.08548527)(127.87911438,93.11209802)(128.13195057,93.11209802)
\curveto(128.8146083,93.11209802)(129.32449463,92.94443765)(129.66160956,92.60911689)
\curveto(129.99872449,92.27379614)(130.16728195,91.76549246)(130.16728195,91.08420585)
\closepath
}
}
{
\newrgbcolor{curcolor}{0 0 0}
\pscustom[linestyle=none,fillstyle=solid,fillcolor=curcolor]
{
\newpath
\moveto(239.74364938,92.55213717)
\lineto(235.6140353,92.55213717)
\curveto(235.67886441,91.32796327)(235.8603859,90.4558929)(236.15859979,89.93592607)
\curveto(236.53028667,89.29722664)(237.03811467,88.97787693)(237.68208379,88.97787693)
\curveto(238.33037486,88.97787693)(238.83388092,89.29927376)(239.19260197,89.94206741)
\curveto(239.50810362,90.50707074)(239.69178609,91.377094)(239.74364938,92.55213717)
\closepath
\moveto(239.72420064,93.59616507)
\curveto(239.60318631,94.75892556)(239.42598675,95.54501716)(239.19260197,95.95443987)
\curveto(238.8209151,96.60132775)(238.31740904,96.92477169)(237.68208379,96.92477169)
\curveto(237.02082691,96.92477169)(236.51515988,96.60542197)(236.1650827,95.96672255)
\curveto(235.88847852,95.45084994)(235.70911799,94.66066411)(235.62700112,93.59616507)
\lineto(239.72420064,93.59616507)
\closepath
\moveto(237.68208379,97.84597278)
\curveto(238.7193495,97.84597278)(239.53403527,97.41198471)(240.1261411,96.54400857)
\curveto(240.71824694,95.68012665)(241.01429986,94.48256523)(241.01429986,92.95132431)
\curveto(241.01429986,91.42417761)(240.71824694,90.22661619)(240.1261411,89.35864005)
\curveto(239.53403527,88.48656968)(238.7193495,88.0505345)(237.68208379,88.0505345)
\curveto(236.64049615,88.0505345)(235.82581038,88.48656968)(235.23802648,89.35864005)
\curveto(234.64592064,90.22661619)(234.34986773,91.42417761)(234.34986773,92.95132431)
\curveto(234.34986773,94.48256523)(234.64592064,95.68012665)(235.23802648,96.54400857)
\curveto(235.82581038,97.41198471)(236.64049615,97.84597278)(237.68208379,97.84597278)
\closepath
}
}
{
\newrgbcolor{curcolor}{0 0 0}
\pscustom[linestyle=none,fillstyle=solid,fillcolor=curcolor]
{
\newpath
\moveto(244.01754708,93.85164484)
\curveto(243.35369703,93.85164484)(242.875725,93.22277156)(242.22515195,93.22277156)
\lineto(242.889002,92.59389828)
\lineto(242.889002,88.82065861)
\curveto(242.889002,88.58168676)(242.76950899,88.51879944)(242.61018498,88.50622197)
\lineto(242.22515195,88.50622197)
\curveto(242.18532094,88.45591211)(242.17204394,88.40560225)(242.17204394,88.35529238)
\curveto(242.15876694,88.30498252)(242.18532094,88.25467266)(242.22515195,88.19178533)
\lineto(244.54862712,88.19178533)
\curveto(244.58845812,88.22951773)(244.61501212,88.29240506)(244.61501212,88.34271492)
\curveto(244.61501212,88.40560225)(244.60173512,88.45591211)(244.54862712,88.50622197)
\lineto(244.15031709,88.50622197)
\curveto(243.99099308,88.51879944)(243.88477707,88.58168676)(243.88477707,88.82065861)
\lineto(243.88477707,92.15368699)
\curveto(244.2698101,92.51843349)(245.06643016,93.22277156)(245.6506182,93.22277156)
\curveto(245.88960422,93.22277156)(246.08875923,93.14730677)(246.22152924,92.95864479)
\curveto(246.31446825,92.89575746)(246.43396126,92.85802506)(246.55345427,92.85802506)
\curveto(246.72605528,92.85802506)(247.0181493,92.98379972)(247.0181493,93.1724617)
\curveto(247.0181493,93.70071525)(246.32774525,93.85164484)(245.88960422,93.85164484)
\curveto(244.97349115,93.85164484)(244.3361951,92.92091239)(243.88477707,92.27946164)
\lineto(243.88477707,93.22277156)
\curveto(243.88477707,93.43658848)(243.93788507,93.65040539)(244.01754708,93.85164484)
\closepath
}
}
{
\newrgbcolor{curcolor}{0 0 0}
\pscustom[linewidth=1,linecolor=curcolor]
{
\newpath
\moveto(265,95.0000067)
\curveto(245,90.0000067)(255,80.0000067)(255,80.0000067)
}
}
{
\newrgbcolor{curcolor}{0 0 0}
\pscustom[linewidth=1,linecolor=curcolor]
{
\newpath
\moveto(110,80.0000067)
\lineto(280,80.0000067)
}
}
{
\newrgbcolor{curcolor}{0 0 0}
\pscustom[linewidth=0.95640802,linecolor=curcolor]
{
\newpath
\moveto(165,123.2705367)
\lineto(165,82.1082867)
}
}
{
\newrgbcolor{curcolor}{0 0 0}
\pscustom[linestyle=none,fillstyle=solid,fillcolor=curcolor]
{
\newpath
\moveto(165,119.44490461)
\lineto(166.91281605,117.53208856)
\lineto(165,124.22694473)
\lineto(163.08718395,117.53208856)
\lineto(165,119.44490461)
\closepath
}
}
{
\newrgbcolor{curcolor}{0 0 0}
\pscustom[linewidth=0.47820401,linecolor=curcolor]
{
\newpath
\moveto(165,119.44490461)
\lineto(166.91281605,117.53208856)
\lineto(165,124.22694473)
\lineto(163.08718395,117.53208856)
\lineto(165,119.44490461)
\closepath
}
}
{
\newrgbcolor{curcolor}{0 0 0}
\pscustom[linestyle=none,fillstyle=solid,fillcolor=curcolor]
{
\newpath
\moveto(165,85.9339188)
\lineto(163.08718395,87.84673484)
\lineto(165,81.15187868)
\lineto(166.91281605,87.84673484)
\lineto(165,85.9339188)
\closepath
}
}
{
\newrgbcolor{curcolor}{0 0 0}
\pscustom[linewidth=0.47820401,linecolor=curcolor]
{
\newpath
\moveto(165,85.9339188)
\lineto(163.08718395,87.84673484)
\lineto(165,81.15187868)
\lineto(166.91281605,87.84673484)
\lineto(165,85.9339188)
\closepath
}
}
{
\newrgbcolor{curcolor}{0 0 0}
\pscustom[linewidth=1,linecolor=curcolor]
{
\newpath
\moveto(110,80.0000067)
\lineto(110,44.9999967)
}
}
{
\newrgbcolor{curcolor}{0 0 0}
\pscustom[linewidth=1,linecolor=curcolor]
{
\newpath
\moveto(280,80.0000067)
\lineto(280,44.9999967)
}
}
{
\newrgbcolor{curcolor}{0 0 0}
\pscustom[linewidth=0.98967332,linecolor=curcolor]
{
\newpath
\moveto(111.74646,55.0000067)
\lineto(278.25352,55.0000067)
}
}
{
\newrgbcolor{curcolor}{0 0 0}
\pscustom[linestyle=none,fillstyle=solid,fillcolor=curcolor]
{
\newpath
\moveto(115.70515327,55.0000067)
\lineto(117.6844999,56.97935333)
\lineto(110.75678668,55.0000067)
\lineto(117.6844999,53.02066007)
\lineto(115.70515327,55.0000067)
\closepath
}
}
{
\newrgbcolor{curcolor}{0 0 0}
\pscustom[linewidth=0.49483666,linecolor=curcolor]
{
\newpath
\moveto(115.70515327,55.0000067)
\lineto(117.6844999,56.97935333)
\lineto(110.75678668,55.0000067)
\lineto(117.6844999,53.02066007)
\lineto(115.70515327,55.0000067)
\closepath
}
}
{
\newrgbcolor{curcolor}{0 0 0}
\pscustom[linestyle=none,fillstyle=solid,fillcolor=curcolor]
{
\newpath
\moveto(274.29482673,55.0000067)
\lineto(272.3154801,53.02066007)
\lineto(279.24319332,55.0000067)
\lineto(272.3154801,56.97935333)
\lineto(274.29482673,55.0000067)
\closepath
}
}
{
\newrgbcolor{curcolor}{0 0 0}
\pscustom[linewidth=0.49483666,linecolor=curcolor]
{
\newpath
\moveto(274.29482673,55.0000067)
\lineto(272.3154801,53.02066007)
\lineto(279.24319332,55.0000067)
\lineto(272.3154801,56.97935333)
\lineto(274.29482673,55.0000067)
\closepath
}
}
{
\newrgbcolor{curcolor}{0 0 0}
\pscustom[linestyle=none,fillstyle=solid,fillcolor=curcolor]
{
\newpath
\moveto(175.08572083,111.07148625)
\curveto(174.08572083,111.07148625)(173.38572083,110.03148625)(172.38572083,110.07148625)
\lineto(173.38572083,109.07148625)
\lineto(173.38572083,97.07148625)
\curveto(173.38572083,96.69148625)(173.20572083,96.59148625)(172.96572083,96.57148625)
\lineto(172.38572083,96.57148625)
\curveto(172.32572083,96.49148625)(172.30572083,96.41148625)(172.30572083,96.33148625)
\curveto(172.28572083,96.25148625)(172.32572083,96.17148625)(172.38572083,96.07148625)
\lineto(175.88572083,96.07148625)
\curveto(175.94572083,96.13148625)(175.98572083,96.23148625)(175.98572083,96.31148625)
\curveto(175.98572083,96.41148625)(175.96572083,96.49148625)(175.88572083,96.57148625)
\lineto(175.28572083,96.57148625)
\curveto(175.04572083,96.59148625)(174.88572083,96.69148625)(174.88572083,97.07148625)
\lineto(174.88572083,102.37148625)
\curveto(175.46572083,102.95148625)(176.66572083,104.07148625)(177.54572083,104.07148625)
\curveto(178.56572083,104.07148625)(178.66572083,102.83148625)(178.66572083,102.07148625)
\lineto(178.66572083,97.07148625)
\curveto(178.66572083,96.69148625)(178.48572083,96.59148625)(178.24572083,96.57148625)
\lineto(177.66572083,96.57148625)
\curveto(177.60572083,96.49148625)(177.58572083,96.41148625)(177.58572083,96.33148625)
\curveto(177.56572083,96.25148625)(177.60572083,96.17148625)(177.66572083,96.07148625)
\lineto(181.16572083,96.07148625)
\curveto(181.22572083,96.13148625)(181.26572083,96.23148625)(181.26572083,96.31148625)
\curveto(181.26572083,96.41148625)(181.24572083,96.49148625)(181.16572083,96.57148625)
\lineto(180.56572083,96.57148625)
\curveto(180.32572083,96.59148625)(180.16572083,96.69148625)(180.16572083,97.07148625)
\lineto(180.16572083,102.57148625)
\curveto(180.16572083,103.89148625)(179.30572083,105.07148625)(177.90572083,105.07148625)
\curveto(176.54572083,105.07148625)(175.56572083,103.57148625)(174.88572083,102.57148625)
\lineto(174.88572083,110.07148625)
\curveto(174.88572083,110.41148625)(174.96572083,110.75148625)(175.08572083,111.07148625)
\closepath
}
}
{
\newrgbcolor{curcolor}{0 0 0}
\pscustom[linestyle=none,fillstyle=solid,fillcolor=curcolor]
{
\newpath
\moveto(194.71427917,72.21430852)
\lineto(195.71427917,71.21430852)
\lineto(195.71427917,66.21430852)
\curveto(195.17427917,66.75430852)(194.05427917,67.21430852)(193.29427917,67.21430852)
\curveto(190.73427917,67.21430852)(189.09427917,64.97430852)(189.09427917,62.61430852)
\curveto(189.09427917,60.25430852)(190.77427917,58.01430852)(193.29427917,58.01430852)
\curveto(194.27427917,58.01430852)(194.89427917,58.21430852)(195.71427917,58.69430852)
\lineto(195.71427917,58.01430852)
\lineto(198.21427917,58.21430852)
\curveto(198.27427917,58.27430852)(198.31427917,58.37430852)(198.31427917,58.45430852)
\curveto(198.31427917,58.55430852)(198.29427917,58.63430852)(198.21427917,58.71430852)
\curveto(197.73427917,58.71430852)(197.21427917,58.55430852)(197.21427917,59.21430852)
\lineto(197.21427917,72.21430852)
\curveto(197.21427917,72.55430852)(197.27427917,72.89430852)(197.41427917,73.21430852)
\curveto(196.41427917,73.21430852)(195.67427917,72.21430852)(194.71427917,72.21430852)
\closepath
\moveto(195.71427917,65.71430852)
\lineto(195.71427917,60.21430852)
\curveto(195.71427917,59.09430852)(194.67427917,58.53430852)(193.67427917,58.53430852)
\curveto(191.25427917,58.53430852)(190.81427917,61.11430852)(190.81427917,63.09430852)
\curveto(190.81427917,64.41430852)(191.79427917,66.69430852)(193.35427917,66.69430852)
\curveto(194.21427917,66.69430852)(195.09427917,66.29430852)(195.71427917,65.71430852)
\closepath
}
}
\end{pspicture}

\end{figure}
\end{document}
