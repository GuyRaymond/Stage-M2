
La tension de surface 

La capillarité étudie l'interface entre l'air et un liquide ou entre 2 liquides
non mixibles.

Le mouillage est l'action de mouiller et mouiller consiste à mettre en contact avec un liquide.

Nous nous interessons en particulier au contact d'une goutte avec un support matériel.

C'est le paramètre d'étalment $S = \gamma_{SG} - \gamma_{SL} - \gamma_{LG}$ qui caractérise le mouillage lorsque la goutte est en équilibre sur un support matériel.

\begin{description}
\item[$S < 0$ :] Mouillage partiel
\item[$S > 0$ :] Mouillage total
\end{description}

Lorsque la que est en équilire nous avons:

\[\gamma_{SG}\ = \gamma_{SL} - \gamma_{LG}\cos\theta_{E} \]
