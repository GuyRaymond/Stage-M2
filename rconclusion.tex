
\section{Conclusion}

Réussir à extraire des paramètres sur des images d'une goutte d'eau dans une soufflerie pour mieux comprendre ce qui se passe quand un goutte glisse sur un plan incliné était l'objectif de notre stage.\\


Pour arriver à atteindre notre objectif, nous avons pris les photos de gouttes d'eau dans la soufflerie soit en essayant d'injecter une goutte de volume donné (avec un certain débit) quand la vitesse était déjà établie dans la soufflerie, soit en injectant d'abord la goutte d'eau de volume donné avant de mettre en marche la souffle pour obtenir la vitesse souhaitée.\\

Les difficultés numériques ont été d'arriver à extraire les paramètres qu'on souhaitait de nos images et celle expérimentale provenait de notre surface qui n'est pas parfaite comme dans la théorie et d'arriver à arriver à une situation de départ où on a le volume de goutte désiré et la vitesse de la soufflerie déjà établie (ou très rapidement établie).

Ce stage a une partie numérique aussi importante que la partie expérimentale

