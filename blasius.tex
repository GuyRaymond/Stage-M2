
L'écoulement peut être séparé entre 2 région dont l'une est la région où les effets visqueux son dominant.

La couche limite est la région dans un écoulement où les effets visqueux 
sont dominant.

La taille caractéristique de la couche limite est souvent petite 
par rappor à la taille caractérisque de.

begin{equation}
 {\partial u\over\partial x}+{\partial \upsilon\over\partial y}=0 
end{equation}

begin{equation}
 u{\partial u \over \partial x}+\upsilon{\partial u \over \partial y}=-{1\over \rho} {\partial p \over \partial x}+{\nu}\left({\partial^2 u\over \partial x^2}+{\partial^2 u\over \partial y^2}\right) 
end{equation}

begin{equation}
 u{\partial \upsilon \over \partial x}+\upsilon{\partial \upsilon \over \partial y}=-{1\over \rho} {\partial p \over \partial y}+{\nu}\left({\partial^2 \upsilon\over \partial x^2}+{\partial^2 \upsilon\over \partial y^2}\right) 
end{equation}


 u{\partial u \over \partial x}+\upsilon{\partial u \over \partial y}=-{1\over \rho} {\partial p \over \partial x}+{\nu}{\partial^2 u\over \partial y^2} 

 {1\over \rho} {\partial p \over \partial y}=0 


 {\partial u\over\partial x}+{\partial \upsilon\over\partial y}=0 

 u{\partial u \over \partial x}+\upsilon{\partial u \over \partial y}=U\frac{dU}{dx}+{\nu}{\partial^2 u\over \partial y^2} 


 \frac{dp}{dx}=0 


 u{\partial u \over \partial x}+\upsilon{\partial u \over \partial y}={\nu}{\partial^2 u\over \partial y^2} 
 

U(x) = U = \text{constant},  d U / d x = 0

x\rightarrow c^2 x, \quad y\rightarrow cy, \quad u\rightarrow u, \quad v\rightarrow \frac{v}{c}


 \eta = \dfrac{y}{\delta(x)} = y \sqrt{ \dfrac{U}{2 \nu x} }, \quad  \psi = \sqrt{2 \nu U x} f(\eta) 

where <math> \delta(x) = \sqrt{  \nu x / U}  is the [[boundary layer thickness]] and <math>\psi is the [[stream function]], in which the newly introduced normalized stream function, <math> f(\eta) , is only a function of the similarity variable. This leads directly to the velocity components

 u(x,y) =   \dfrac{\partial \psi}{\partial y} = U f'(\eta), \quad v(x,y) = - \dfrac{\partial \psi}{\partial x} = \sqrt{\dfrac{\nu U}{2 x}} [ \eta f'(\eta) - f(\eta)]  

Where the prime denotes derivation with respect to <math> \eta .
Substitution into the momentum equation gives the Blasius equation

 f''' + f'' f = 0 

The boundary conditions are the [[no-slip condition]], the impermeability of the wall and the free stream velocity outside the boundary layer

 u(x,0) = 0 \rightarrow f'(0) = 0 
 v(x,0) = 0 \rightarrow f(0) = 0  
 u(x,\infty) = U \rightarrow f'(\infty) = 1 
