\documentclass[french]{article}
 
\usepackage[utf8]{inputenc}
\usepackage[T1]{fontenc}
\usepackage{babel}
\usepackage{amsmath} 
\begin{document}
 
\tableofcontents
 
\vspace{2cm} %Add a 2cm space
 
\begin{abstract}
We are studied a drop of water sliding on an horizontal plan. The drop was placed inside a wind tunnel.
\end{abstract}

\section{Couche limite}
Notre goutte d'eau (de volume de l'ordre millilitre) est sur une paroi horizontales en présence d'un écoulement d'air.


Pour déterminer les vitesses de l'écoulement d'air au voisinage de notre goutte d'eau, nous devons nous intéresser à la couche limite puisque l'écoulement d'un fluide en présence d'une paroi peut être séparé en deux régions dont l'une proche de la paroi où les effets visqueux ne peuvent être négligés par rapport aux effets inertiels et l'autre région à l'extérieur de la précédente (avec une frontière commune) où les effets visqueux peuvent être négligés.

La couche limite est cette région proche de la paroi où les effets visqueux ne peuvent être négligés.

C'est Prandtl qui fut le premier à definir la couche limite. 

\subsection{Equation de la couche limite }
Les hypothèses faites sont : un écoulement stationnaire, incompressible, bidimensionnelle et uniforme à l'infini (loin de la paroi).

Notre paroi est placée en $y = 0$

Nous rajoutons l'hypthosèse d'écoulement laminaire puisque notre écoulement était laminaire laminaire.

l'équation de la continuté s'écrit alors:

\begin{equation}
\frac{\partial u}{\partial w}
\end{equation}

\begin{equation}
\label{Eq:1}
\frac{\partial u}{\partial x} + \frac{\partial v}{\partial y} = 0
\end{equation}
%l'équation de la continuté s'écrit alors:
%
%Celle de
%Navier-Stockes s'écrit:
%\begin{equation}[Navier, Stockes]
%\label{Eq:Stockes}
%	u\frac{\partial u}{\partial x} + v\frac{\partial u}{\partial y} = - \frac{1}{\rho}+ \nu\left(\frac{\partial^{2} u}{x^{2}} + \frac{\partial^{2} u}{y^{2}}\right)
%	u\frac{\partial v}{\partial x} + v\frac{\partial v}{\partial y} = - \frac{1}{\rho}+ \nu\left(\frac{\partial^{2} v}{x^{2}} + \frac{\partial^{2} v}{y^{2}}\right)
%\end{equation}
%\subsection{Equation de Blasius}
%\begin{thebibliography}{}
%\bibitem{RefJ}
%% Format for Journal Reference
%Author, Article title, Journal, Volume, page numbers (year)
%% Format for books
%\bibitem{RefB}
%Author, Book title, page numbers. Publisher, place (year)
%\end{thebibliography}
\end{document}


