\documentclass[french]{article}
 
\usepackage[utf8]{inputenc}
\usepackage[T1]{fontenc}
\usepackage{babel}
\usepackage{amsmath} 
\begin{document}
 
\tableofcontents
 
\vspace{2cm} %Add a 2cm space
 
\begin{abstract}
We are studied a drop of water sliding on an horizontal plan. The drop was placed inside a wind tunnel.
\end{abstract}

\section{Couche limite}
Notre goutte d'eau (de volume de l'ordre millilitre) est sur une paroi horizontales en présence d'un écoulement d'air.


Pour déterminer les vitesses de l'écoulement d'air au voisinage de notre goutte d'eau, nous devons nous intéresser à la couche limite puisque l'écoulement d'un fluide en présence d'une paroi peut être séparé en deux régions dont l'une proche de la paroi où les effets visqueux ne peuvent être négligés par rapport aux effets inertiels et l'autre région à l'extérieur de la précédente (avec une frontière commune) où les effets visqueux peuvent être négligés.

La couche limite est cette région proche de la paroi où les effets visqueux ne peuvent être négligés.

C'est Prandtl qui fut le premier à definir la couche limite. 

\subsection{Equation de la couche limite }

Les équations de la couche limite sont définies pour un écoulement bidimensionnel et nous nous les donnerons uniquement pour le cas d'un écoulement laminaire puisque nos écoulements étaient laminaires dans nos expériences.

Soit $u$ la vitesse de l'écoulement parallèle à la paroi suivant l'axe $x$ et $v$ la vitesse normale à la paroi suivant l'axe $y$.

Soit $U$ et $V$ les vitesses caractéristiques dans la couche limite respectivement $u$ et l'axe $v$.

Soit $L$ et $\delta$ la taille caractéristique de la couche limite suivant respectivement l'axe $x$ et l'axe $y$.

En plus de l'hypothèse d'écoulement bidimensionnel, les hypothèses faites sont : écoulement stationnaire, incompressible, $\delta << L$ et les effets visqueux et inertiels ont du même ordre de grandeur dans la couche limite.



l'équation de la continuté s'écrit alors:

\begin{equation}
\label{Eq:1}
\frac{\partial u}{\partial x} + \frac{\partial v}{\partial y} = 0
\end{equation}

De cette équation : $V \approx \frac{U\delta}{L} $



L'équation de Navier-Stokes s'écrit:

\begin{align}[Navier, Stokes]
u\frac{\partial u}{\partial x} + v\frac{\partial u}{\partial y} &= - \frac{1}{\rho}\frac{\partial p}{\partial  x} + \nu\left(\frac{\partial^{2} u}{\partial  x^{2}} + \frac{\partial^{2} u}{\partial  y^{2}}\right) \\

u\frac{\partial v}{\partial x} + v\frac{\partial v}{\partial y} &= - \frac{1}{\rho}\frac{\partial p}{\partial } + \nu\left(\frac{\partial^{2} v}{\partial  x^{2}} + \frac{\partial^{2} v}{\partial  y^{2}}\right)
\end{align}


%\subsection{Equation de Blasius}
%\begin{thebibliography}{}
%\bibitem{RefJ}
%% Format for Journal Reference
%Author, Article title, Journal, Volume, page numbers (year)
%% Format for books
%\bibitem{RefB}
%Author, Book title, page numbers. Publisher, place (year)
%\end{thebibliography}
\end{document}


