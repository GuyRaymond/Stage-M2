
\section{Introduction}

Dans le domaine des transports, avoir des gouttes de pluie sur le pare brise d'une voiture est courant et ces gouttes peuvent ruisseler sous l'action du vent.
Dans le cas de l'aéronautique, les gouttes d'eau qui ne ruissellent pas, qui ne glissent pas, peuvent givrer (devenir de petits morceaux de glace) et nuire au bon fonctionnement de l'appareil.\\

L'objectif de notre stage est d'étudier le glissement d'une goutte d'eau sur un plaque plane et d'établir le lien entre les paramètres qui régissent ce phénomène de glissement comme la taille de la goutte à partir de laquelle il y a glissement ou le lien des angles de contact avec le glissement.\\

Pour atteindre notre objectif, nous prendrons les images (à l'aide d'une caméra) de gouttes d'eau glissant sur une plaque plane dans une soufflerie et nous traiterons ensuite ces images à l'aide de Matlab pour obtenir des paramètres de la courbe pendant son glissement, puis nous chercherons à interpréter ces données.
